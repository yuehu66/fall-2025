\documentclass[12pt,english]{article}
\usepackage{mathptmx}
\usepackage[utf8]{inputenc}
\usepackage{babel}
\usepackage{geometry}
\usepackage{xcolor}
\geometry{verbose,tmargin=1in,bmargin=1in,lmargin=1in,rmargin=1in}
\usepackage{amsmath}
\usepackage{amsthm}
\usepackage{amssymb}
\usepackage[authoryear]{natbib}
\usepackage{minted}
\usepackage{mathtools}
\definecolor{bg}{rgb}{0.95,0.95,0.95}
\usepackage[unicode=true,pdfusetitle,
 bookmarks=true,bookmarksnumbered=false,bookmarksopen=false,
 breaklinks=false,pdfborder={0 0 0},pdfborderstyle={},backref=false,colorlinks=false]
 {hyperref}
\usepackage{breakurl}

\begin{document}

\title{In-Class Activity: Machine Learning Basics\\
Bias-Variance Tradeoff with Basis Function Approximation}
\author{ECON 6343: Structural Econometrics\\
Prof. Tyler Ransom\\
University of Oklahoma}
\date{}
\maketitle

\section*{Overview}
Today we'll explore fundamental machine learning concepts through hands-on coding with a high-dimensional problem. We'll generate data from a sparse nonlinear process using basis function approximation, then demonstrate how proper ML techniques (train/validation/test splitting, regularization) help us recover the true relationship while avoiding overfitting in a $p \approx n$ setting.

\subsection*{Learning Objectives}
\begin{itemize}
    \item Understand the bias-variance tradeoff empirically in high dimensions
    \item Practice train/validation/test data splitting
    \item Work with polynomial basis function approximation
    \item Compare OLS, Ridge, and LASSO regularization
    \item Evaluate out-of-sample prediction performance
    \item Visualize overfitting, underfitting, and sparsity selection
\end{itemize}

\subsection*{Required Packages}
\begin{minted}[bgcolor=bg]{julia}
using Random, LinearAlgebra, Statistics, Optim
using DataFrames, Plots, GLM, StatsPlots
using MLJ, MLJLinearModels, Tables
\end{minted}

\section{Understanding the Problem Setup (10 minutes)}

\subsection{The High-Dimensional Challenge}

We'll work with a problem that mimics many real-world econometric applications:
\begin{itemize}
    \item \textbf{Raw features:} 2 variables ($x_1, x_2$)
    \item \textbf{Basis expansion:} Transform into 120 polynomial basis functions (up to degree 4, including interactions)
    \item \textbf{True sparsity:} Only 5 of the 120 basis coefficients are actually non-zero
    \item \textbf{Sample size:} 300 total observations (split into train/val/test)
    \item \textbf{Challenge:} $p \approx n$ after train split (120 features, $\sim$180 training obs)
\end{itemize}

This creates a realistic scenario where:
\begin{itemize}
    \item The true relationship is sparse (most basis terms don't matter)
    \item We don't know \textit{which} basis terms matter
    \item OLS will severely overfit
    \item Regularization is essential
\end{itemize}

\subsection{Polynomial Basis Functions}

A polynomial basis transforms raw features into higher-order terms:
\begin{align*}
\phi(x_1, x_2) = [x_1, x_2, x_1^2, x_2^2, x_1^3, x_2^3, x_1^4, x_2^4, x_1x_2, x_1^2x_2, \ldots]
\end{align*}

The true model in our data generation is:
\begin{align*}
y = \sum_{j=1}^{5} \beta_j \phi_j(x_1, x_2) + \epsilon
\end{align*}

where only 5 of the 120 $\beta$ coefficients are non-zero.

\section{Utility Functions (15 minutes)}

\subsection{Basis Function Generation}

\begin{minted}[bgcolor=bg]{julia}
"""
Generate polynomial basis for multivariate input
"""
function polynomial_basis(X::Matrix{Float64}, max_degree::Int)
    n, d = size(X)
    bases = []
    
    # Univariate polynomials for each feature
    for j in 1:d
        for deg in 1:max_degree
            push!(bases, X[:, j].^deg)
        end
    end
    
    # Interactions (if multiple features)
    if d > 1
        for j1 in 1:d, j2 in (j1+1):d
            for deg1 in 1:max_degree, deg2 in 1:max_degree
                if deg1 + deg2 <= max_degree
                    push!(bases, X[:, j1].^deg1 .* X[:, j2].^deg2)
                end
            end
        end
    end
    
    return hcat(bases...)
end
\end{minted}

\textbf{Task:} What happens to the number of basis functions as we increase \texttt{max\_degree}? Calculate for $d=2$ and degrees 2, 3, 4.

\subsection{Data Generation with Sparsity}

\begin{minted}[bgcolor=bg]{julia}
"""
Generate data with sparse nonlinear structure
"""
function generate_data(n::Int, n_features::Int, n_basis::Int, 
                      n_active::Int; noise_std=0.5)
    # Generate raw features
    X = randn(n, n_features)
    
    # Create basis expansion (up to 4th degree)
    Phi = polynomial_basis(X, 4)
    
    # Truncate/pad to exactly n_basis functions
    if size(Phi, 2) > n_basis
        Phi = Phi[:, 1:n_basis]
    elseif size(Phi, 2) < n_basis
        extra = n_basis - size(Phi, 2)
        Phi = hcat(Phi, randn(n, extra))
    end
    
    # True coefficients: only first n_active are non-zero
    beta_true = zeros(n_basis)
    beta_true[1:n_active] = randn(n_active) .* 6.0
    
    # Generate outcome
    y = Phi * beta_true + randn(n) * noise_std
    
    return X, Phi, y, beta_true
end
\end{minted}

\subsection{Train/Validation/Test Split}

\begin{minted}[bgcolor=bg]{julia}
"""
Split data into train/val/test
"""
function split_data(Phi, y; train_frac=0.6, val_frac=0.2)
    n = length(y)
    n_train = Int(floor(train_frac * n))
    n_val = Int(floor(val_frac * n))
    
    idx = Random.shuffle(1:n)
    train_idx = idx[1:n_train]
    val_idx = idx[n_train+1:n_train+n_val]
    test_idx = idx[n_train+n_val+1:end]
    
    return (
        train = (Phi[train_idx, :], y[train_idx]),
        val = (Phi[val_idx, :], y[val_idx]),
        test = (Phi[test_idx, :], y[test_idx]),
        idx = (train=train_idx, val=val_idx, test=test_idx)
    )
end
\end{minted}

\textbf{Key principle:} The test set remains locked away until final evaluation!

\subsection{Regression Functions}

\begin{minted}[bgcolor=bg]{julia}
"""
Fit OLS regression
"""
function fit_ols(X::Matrix{Float64}, y::Vector{Float64})
    X_design = hcat(ones(size(X, 1)), X)
    beta = (X_design' * X_design) \ (X_design' * y)
    return beta
end

"""
Predict with OLS coefficients
"""
function predict_ols(X::Matrix{Float64}, beta::Vector{Float64})
    X_design = hcat(ones(size(X, 1)), X)
    return X_design * beta
end

"""
Fit Ridge regression (L2 penalty)
"""
function fit_ridge(X::Matrix{Float64}, y::Vector{Float64}, lambda::Float64)
    n, p = size(X)
    X_design = hcat(ones(n), X)
    # Don't penalize intercept
    penalty = diagm([0.0; fill(lambda, p)])
    beta = (X_design' * X_design + penalty) \ (X_design' * y)
    return beta
end

"""
Compute MSE
"""
function compute_mse(y_true::Vector{Float64}, y_pred::Vector{Float64})
    return mean((y_true .- y_pred).^2)
end
\end{minted}

\section{Data Generation and Preprocessing (10 minutes)}

\begin{minted}[bgcolor=bg]{julia}
Random.seed!(1234)

# Parameters
n_total = 300
n_features = 2      # Raw features
n_basis = 120       # Basis functions (p)
n_active = 5       # True non-zero coefficients

println("Data Generation:")
println("  Total observations: ", n_total)
println("  Raw features: ", n_features)
println("  Basis functions (p): ", n_basis)
println("  True active coefficients: ", n_active)

# Generate data
X, Phi, y, beta_true = generate_data(n_total, n_features, n_basis, 
                                n_active, noise_std=23)

# Split data
data = split_data(Phi, y)
Phi_train, y_train = data.train
Phi_val, y_val = data.val
Phi_test, y_test = data.test

println("\nData Split:")
println("  Training: ", length(y_train), " obs")
println("  Validation: ", length(y_val), " obs")
println("  Test: ", length(y_test), " obs")
println("  p/n ratio: ", round(n_basis/length(y_train), digits=2))
\end{minted}

\subsection{Standardization}

Critical for regularization methods:

\begin{minted}[bgcolor=bg]{julia}
# Standardize based on training set
mu = mean(Phi_train, dims=1)
sigma = std(Phi_train, dims=1)
Phi_train = (Phi_train .- mu) ./ sigma
Phi_val = (Phi_val .- mu) ./ sigma
Phi_test = (Phi_test .- mu) ./ sigma

# Standardize y
y_mean = mean(y_train)
y_std = std(y_train)
y_train = (y_train .- y_mean) ./ y_std
y_val = (y_val .- y_mean) ./ y_std
y_test = (y_test .- y_mean) ./ y_std
\end{minted}

\textbf{Discussion:} Why standardize? Why use training set statistics?

\section{Ordinary Least Squares (10 minutes)}

Let's see what happens with OLS in a high-dimensional setting:

\begin{minted}[bgcolor=bg]{julia}
# Fit OLS
beta_ols = fit_ols(Phi_train, y_train)

# Compute MSE
mse_train_ols = compute_mse(y_train, predict_ols(Phi_train, beta_ols))
mse_val_ols = compute_mse(y_val, predict_ols(Phi_val, beta_ols))

println("\nOLS Results:")
println("  Training MSE: ", round(mse_train_ols, digits=4))
println("  Validation MSE: ", round(mse_val_ols, digits=4))
println("  Val/Train ratio: ", round(mse_val_ols/mse_train_ols, digits=2), "x")
\end{minted}

\textbf{Expected outcome:} Severe overfitting! The validation MSE should be much larger than training MSE.

\textbf{Task:} 
\begin{itemize}
    \item What does the val/train ratio tell us?
    \item Why does OLS perform poorly here?
    \item What's the relationship between $p$, $n$, and overfitting?
\end{itemize}

\section{Ridge Regression (L2 Regularization) (15 minutes)}

Ridge adds a penalty on the sum of squared coefficients:
\begin{align*}
\min_\beta \sum_{i=1}^n (y_i - \phi_i'\beta)^2 + \lambda\sum_{j=1}^p \beta_j^2
\end{align*}

\subsection{Cross-Validation for $\lambda$}

\begin{minted}[bgcolor=bg]{julia}
"""
Cross-validate to find optimal lambda for Ridge
"""
function cv_ridge(Phi_train, y_train, Phi_val, y_val, lambda_grid)
    mse_train = zeros(length(lambda_grid))
    mse_val = zeros(length(lambda_grid))
    
    for (i, lambda) in enumerate(lambda_grid)
        beta = fit_ridge(Phi_train, y_train, lambda)
        mse_train[i] = compute_mse(y_train, predict_ols(Phi_train, beta))
        mse_val[i] = compute_mse(y_val, predict_ols(Phi_val, beta))
    end
    
    best_idx = argmin(mse_val)
    return lambda_grid[best_idx], mse_train, mse_val, best_idx
end

# Run cross-validation
lambda_grid = exp.(range(log(1e-6), log(1000), length=500))
lambda_best_ridge, mse_train_ridge, mse_val_ridge, best_idx_ridge = 
    cv_ridge(Phi_train, y_train, Phi_val, y_val, lambda_grid)

println("\nRidge Regression Results:")
println("  Best lambda: ", round(lambda_best_ridge, digits=4))
println("  Training MSE: ", round(mse_train_ridge[best_idx_ridge], digits=4))
println("  Validation MSE: ", round(mse_val_ridge[best_idx_ridge], digits=4))
println("  Val/Train ratio: ", 
        round(mse_val_ridge[best_idx_ridge]/mse_train_ridge[best_idx_ridge], 
              digits=2), "x")
println("  Improvement over OLS: ", 
        round(100*(1 - mse_val_ridge[best_idx_ridge]/mse_val_ols), digits=1), "%")
\end{minted}

\textbf{Discussion:}
\begin{itemize}
    \item What happens as $\lambda \to 0$? As $\lambda \to \infty$?
    \item Why does Ridge help with overfitting?
    \item Limitation: Ridge doesn't set coefficients exactly to zero
\end{itemize}

\section{LASSO Regression (L1 Regularization) (15 minutes)}

LASSO adds a penalty on the sum of absolute values:
\begin{align*}
\min_\beta \sum_{i=1}^n (y_i - \phi_i'\beta)^2 + \lambda\sum_{j=1}^p |\beta_j|
\end{align*}

Key advantage: LASSO performs \textbf{variable selection} by setting some coefficients exactly to zero.

\subsection{Cross-Validation with Sparsity Tracking}

\begin{minted}[bgcolor=bg]{julia}
"""
Cross-validate LASSO
"""
function cv_lasso(Phi_train, y_train, Phi_val, y_val, lambda_grid)
    mse_train = zeros(length(lambda_grid))
    mse_val = zeros(length(lambda_grid))
    n_nonzero = zeros(Int, length(lambda_grid))
    
    Phi_train_tbl = Tables.table(Phi_train)
    Phi_val_tbl = Tables.table(Phi_val)
    
    for (i, lambda) in enumerate(lambda_grid)
        lasso = LassoRegressor(lambda=lambda, solver=ProxGrad(max_iter=10_000))
        mach = machine(lasso, Phi_train_tbl, y_train)
        fit!(mach, verbosity=0)
        
        y_train_pred = MLJ.predict(mach, Phi_train_tbl)
        y_val_pred = MLJ.predict(mach, Phi_val_tbl)
        
        mse_train[i] = compute_mse(y_train, y_train_pred)
        mse_val[i] = compute_mse(y_val, y_val_pred)
        
        beta = fitted_params(mach).coefs
        n_nonzero[i] = sum(abs.(last.(beta)) .> 1e-6)
    end
    
    best_idx = argmin(mse_val)
    return lambda_grid[best_idx], mse_train, mse_val, n_nonzero, best_idx
end

# Run cross-validation
lambda_grid_lasso = exp.(range(log(1e-6), log(1000), length=500))
lambda_best_lasso, mse_train_lasso, mse_val_lasso, n_nonzero, best_idx_lasso = 
    cv_lasso(Phi_train, y_train, Phi_val, y_val, lambda_grid_lasso)

println("\nLASSO Regression Results:")
println("  Best lambda: ", round(lambda_best_lasso, digits=4))
println("  Training MSE: ", round(mse_train_lasso[best_idx_lasso], digits=4))
println("  Validation MSE: ", round(mse_val_lasso[best_idx_lasso], digits=4))
println("  Selected features: ", n_nonzero[best_idx_lasso], " of ", n_basis)
println("  True active features: ", n_active)
println("  Improvement over OLS: ", 
        round(100*(1 - mse_val_lasso[best_idx_lasso]/mse_val_ols), digits=1), "%")
\end{minted}

\textbf{Discussion:}
\begin{itemize}
    \item How many features did LASSO select vs. the true number (5)?
    \item Why is sparsity desirable for interpretation?
    \item When would you prefer LASSO over Ridge?
\end{itemize}

\section{Visualization (15 minutes)}

\subsection{Ridge Regularization Path}

\begin{minted}[bgcolor=bg]{julia}
# Plot 1: Ridge regularization path
p1 = plot(log.(lambda_grid), [mse_train_ridge mse_val_ridge],
     label=["Training MSE" "Validation MSE"],
     xlabel="log(lambda)", ylabel="MSE",
     title="Ridge: Bias-Variance Tradeoff",
     linewidth=2, legend=:topleft,
     size=(800, 500))
vline!(p1, [log(lambda_best_ridge)], label="Optimal lambda", 
       linestyle=:dash, linewidth=2, color=:red)
savefig(p1, "ridge_regularization_path.png")
\end{minted}

\textbf{What to look for:}
\begin{itemize}
    \item Training MSE increases with $\lambda$ (more bias)
    \item Validation MSE is U-shaped (bias-variance tradeoff)
    \item Optimal $\lambda$ minimizes validation error
\end{itemize}

\subsection{LASSO Sparsity Path}

\begin{minted}[bgcolor=bg]{julia}
# Plot 2: LASSO sparsity path
p2 = plot(n_nonzero, [mse_train_lasso mse_val_lasso],
     label=["Training MSE" "Validation MSE"],
     xlabel="Number of Selected Features", ylabel="MSE",
     title="LASSO: Model Complexity vs Performance",
     linewidth=2, legend=:topright,
     size=(800, 500))
vline!(p2, [n_nonzero[best_idx_lasso]], label="Optimal", 
       linestyle=:dash, linewidth=2, color=:red)
vline!(p2, [n_active], label="True # Active", 
       linestyle=:dot, linewidth=2, color=:green)
savefig(p2, "lasso_sparsity_path.png")
\end{minted}

\subsection{Method Comparison}

\begin{minted}[bgcolor=bg]{julia}
# Create results table
results = DataFrame(
    Method = ["OLS", "Ridge", "LASSO"],
    Train_MSE = [mse_train_ols, 
                 mse_train_ridge[best_idx_ridge], 
                 mse_train_lasso[best_idx_lasso]],
    Val_MSE = [mse_val_ols, 
               mse_val_ridge[best_idx_ridge], 
               mse_val_lasso[best_idx_lasso]]
)

# Plot 3: Method comparison
p3 = groupedbar([results.Train_MSE results.Val_MSE],
           bar_position=:dodge,
           label=["Train" "Validation"],
           xlabel="Method", ylabel="MSE",
           title="Performance Comparison Across Data Splits",
           xticks=(1:3, results.Method),
           legend=:topright,
           size=(800, 500))
savefig(p3, "method_comparison.png")
\end{minted}

\section{Final Test Set Evaluation (10 minutes)}

Only now do we evaluate on the test set!

\begin{minted}[bgcolor=bg]{julia}
# Refit with best hyperparameters
beta_ridge_final = fit_ridge(Phi_train, y_train, lambda_best_ridge)

lasso_final = LassoRegressor(lambda=lambda_best_lasso)
mach_final = machine(lasso_final, Tables.table(Phi_train), y_train)
fit!(mach_final, verbosity=0)

# Test predictions
mse_test_ols = compute_mse(y_test, predict_ols(Phi_test, beta_ols))
mse_test_ridge = compute_mse(y_test, predict_ols(Phi_test, beta_ridge_final))
mse_test_lasso = compute_mse(y_test, 
                              MLJ.predict(mach_final, Tables.table(Phi_test)))

println("\n" * "="^50)
println("FINAL TEST SET RESULTS")
println("="^50)
println("OLS Test MSE:    ", round(mse_test_ols, digits=4))
println("Ridge Test MSE:  ", round(mse_test_ridge, digits=4))
println("LASSO Test MSE:  ", round(mse_test_lasso, digits=4))

# Add to results table
results.Test_MSE = [mse_test_ols, mse_test_ridge, mse_test_lasso]
results.Overfit_Ratio = [mse_val_ols/mse_train_ols,
                         mse_val_ridge[best_idx_ridge]/mse_train_ridge[best_idx_ridge],
                         mse_val_lasso[best_idx_lasso]/mse_train_lasso[best_idx_lasso]]

println("\n", results)
\end{minted}

\textbf{Critical question:} Does test MSE match validation MSE? If not, what went wrong?

\section{Key Takeaways (10 minutes)}

\subsection{Main Lessons}

\begin{enumerate}
\item \textbf{High-Dimensional Problems ($p \approx n$):}
    \begin{itemize}
        \item OLS fails catastrophically when $p$ is close to $n$
        \item Training error becomes misleading (can be arbitrarily small)
        \item Regularization is not optional—it's essential
    \end{itemize}

\item \textbf{Bias-Variance Tradeoff:}
    \begin{itemize}
        \item No regularization ($\lambda=0$): High variance, overfits training noise
        \item Too much regularization ($\lambda \to \infty$): High bias, underfits signal
        \item Optimal $\lambda$: Balances bias and variance for best generalization
    \end{itemize}

\item \textbf{Train/Validation/Test Split:}
    \begin{itemize}
        \item Training: Estimate model parameters ($\beta$)
        \item Validation: Choose hyperparameters ($\lambda$)
        \item Test: Assess final performance (never touch until end!)
    \end{itemize}

\item \textbf{Regularization Methods:}
    \begin{itemize}
        \item Ridge (L2): Shrinks all coefficients, works well with correlated features
        \item LASSO (L1): Sparse selection, sets many coefficients to exactly zero
        \item Choice depends on whether you believe in sparsity
    \end{itemize}

\item \textbf{Sparsity and Interpretability:}
    \begin{itemize}
        \item True model: Only 5 of 120 basis terms are active
        \item LASSO automatically discovers relevant features
        \item Sparse models are more interpretable and computationally efficient
    \end{itemize}
\end{enumerate}

\subsection{Connection to Econometrics}

\paragraph{When do economists face high-dimensional problems?}
\begin{itemize}
    \item Many control variables relative to sample size
    \item Flexible functional forms (polynomials, splines, interactions)
    \item Fixed effects with many categories
    \item Instrumental variables with many instruments (weak IV problem)
    \item Time series with many lags
\end{itemize}


\subsection{Extensions to Explore}

\begin{itemize}
\item \textbf{Elastic Net:} Combine L1 and L2 penalties: $\lambda_1|\beta| + \lambda_2\beta^2$
\item \textbf{$k$-Fold Cross-Validation:} Use all data more efficiently for hyperparameter tuning
\item \textbf{Different Basis Functions:} Try Chebyshev, B-splines, wavelets
\item \textbf{Classification:} Binary outcomes with regularized logistic regression
\item \textbf{Tree-Based Methods:} Random Forest, XGBoost for comparison
\item \textbf{Neural Networks:} Deep learning as ultimate flexible basis expansion
\end{itemize}

\subsection{Implementation Tips}

\begin{itemize}
\item Always standardize features before applying regularization
\item Use a fine grid for $\lambda$ (logarithmic spacing works well)
\item Monitor both training and validation error to diagnose bias vs. variance
\item For LASSO, track sparsity to understand model selection
\item Never tune hyperparameters using test set performance
\item In practice, use $k$-fold CV instead of single validation set
\end{itemize}

\subsection{Theoretical Foundations}

For those interested in the theory:
\begin{itemize}
\item Ridge solution: $\hat{\beta}_{Ridge} = (X'X + \lambda I)^{-1}X'y$
\item LASSO has no closed form, requires iterative algorithms (coordinate descent, LARS)
\item Oracle inequality: Regularization achieves optimal prediction rate under sparsity
\end{itemize}

\end{document}